% Options for packages loaded elsewhere
\PassOptionsToPackage{unicode}{hyperref}
\PassOptionsToPackage{hyphens}{url}
%
\documentclass[
]{article}
\usepackage{amsmath,amssymb}
\usepackage{iftex}
\ifPDFTeX
  \usepackage[T1]{fontenc}
  \usepackage[utf8]{inputenc}
  \usepackage{textcomp} % provide euro and other symbols
\else % if luatex or xetex
  \usepackage{unicode-math} % this also loads fontspec
  \defaultfontfeatures{Scale=MatchLowercase}
  \defaultfontfeatures[\rmfamily]{Ligatures=TeX,Scale=1}
\fi
\usepackage{lmodern}
\ifPDFTeX\else
  % xetex/luatex font selection
\fi
% Use upquote if available, for straight quotes in verbatim environments
\IfFileExists{upquote.sty}{\usepackage{upquote}}{}
\IfFileExists{microtype.sty}{% use microtype if available
  \usepackage[]{microtype}
  \UseMicrotypeSet[protrusion]{basicmath} % disable protrusion for tt fonts
}{}
\makeatletter
\@ifundefined{KOMAClassName}{% if non-KOMA class
  \IfFileExists{parskip.sty}{%
    \usepackage{parskip}
  }{% else
    \setlength{\parindent}{0pt}
    \setlength{\parskip}{6pt plus 2pt minus 1pt}}
}{% if KOMA class
  \KOMAoptions{parskip=half}}
\makeatother
\usepackage{xcolor}
\usepackage[margin=1in]{geometry}
\usepackage{color}
\usepackage{fancyvrb}
\newcommand{\VerbBar}{|}
\newcommand{\VERB}{\Verb[commandchars=\\\{\}]}
\DefineVerbatimEnvironment{Highlighting}{Verbatim}{commandchars=\\\{\}}
% Add ',fontsize=\small' for more characters per line
\usepackage{framed}
\definecolor{shadecolor}{RGB}{248,248,248}
\newenvironment{Shaded}{\begin{snugshade}}{\end{snugshade}}
\newcommand{\AlertTok}[1]{\textcolor[rgb]{0.94,0.16,0.16}{#1}}
\newcommand{\AnnotationTok}[1]{\textcolor[rgb]{0.56,0.35,0.01}{\textbf{\textit{#1}}}}
\newcommand{\AttributeTok}[1]{\textcolor[rgb]{0.13,0.29,0.53}{#1}}
\newcommand{\BaseNTok}[1]{\textcolor[rgb]{0.00,0.00,0.81}{#1}}
\newcommand{\BuiltInTok}[1]{#1}
\newcommand{\CharTok}[1]{\textcolor[rgb]{0.31,0.60,0.02}{#1}}
\newcommand{\CommentTok}[1]{\textcolor[rgb]{0.56,0.35,0.01}{\textit{#1}}}
\newcommand{\CommentVarTok}[1]{\textcolor[rgb]{0.56,0.35,0.01}{\textbf{\textit{#1}}}}
\newcommand{\ConstantTok}[1]{\textcolor[rgb]{0.56,0.35,0.01}{#1}}
\newcommand{\ControlFlowTok}[1]{\textcolor[rgb]{0.13,0.29,0.53}{\textbf{#1}}}
\newcommand{\DataTypeTok}[1]{\textcolor[rgb]{0.13,0.29,0.53}{#1}}
\newcommand{\DecValTok}[1]{\textcolor[rgb]{0.00,0.00,0.81}{#1}}
\newcommand{\DocumentationTok}[1]{\textcolor[rgb]{0.56,0.35,0.01}{\textbf{\textit{#1}}}}
\newcommand{\ErrorTok}[1]{\textcolor[rgb]{0.64,0.00,0.00}{\textbf{#1}}}
\newcommand{\ExtensionTok}[1]{#1}
\newcommand{\FloatTok}[1]{\textcolor[rgb]{0.00,0.00,0.81}{#1}}
\newcommand{\FunctionTok}[1]{\textcolor[rgb]{0.13,0.29,0.53}{\textbf{#1}}}
\newcommand{\ImportTok}[1]{#1}
\newcommand{\InformationTok}[1]{\textcolor[rgb]{0.56,0.35,0.01}{\textbf{\textit{#1}}}}
\newcommand{\KeywordTok}[1]{\textcolor[rgb]{0.13,0.29,0.53}{\textbf{#1}}}
\newcommand{\NormalTok}[1]{#1}
\newcommand{\OperatorTok}[1]{\textcolor[rgb]{0.81,0.36,0.00}{\textbf{#1}}}
\newcommand{\OtherTok}[1]{\textcolor[rgb]{0.56,0.35,0.01}{#1}}
\newcommand{\PreprocessorTok}[1]{\textcolor[rgb]{0.56,0.35,0.01}{\textit{#1}}}
\newcommand{\RegionMarkerTok}[1]{#1}
\newcommand{\SpecialCharTok}[1]{\textcolor[rgb]{0.81,0.36,0.00}{\textbf{#1}}}
\newcommand{\SpecialStringTok}[1]{\textcolor[rgb]{0.31,0.60,0.02}{#1}}
\newcommand{\StringTok}[1]{\textcolor[rgb]{0.31,0.60,0.02}{#1}}
\newcommand{\VariableTok}[1]{\textcolor[rgb]{0.00,0.00,0.00}{#1}}
\newcommand{\VerbatimStringTok}[1]{\textcolor[rgb]{0.31,0.60,0.02}{#1}}
\newcommand{\WarningTok}[1]{\textcolor[rgb]{0.56,0.35,0.01}{\textbf{\textit{#1}}}}
\usepackage{graphicx}
\makeatletter
\def\maxwidth{\ifdim\Gin@nat@width>\linewidth\linewidth\else\Gin@nat@width\fi}
\def\maxheight{\ifdim\Gin@nat@height>\textheight\textheight\else\Gin@nat@height\fi}
\makeatother
% Scale images if necessary, so that they will not overflow the page
% margins by default, and it is still possible to overwrite the defaults
% using explicit options in \includegraphics[width, height, ...]{}
\setkeys{Gin}{width=\maxwidth,height=\maxheight,keepaspectratio}
% Set default figure placement to htbp
\makeatletter
\def\fps@figure{htbp}
\makeatother
\setlength{\emergencystretch}{3em} % prevent overfull lines
\providecommand{\tightlist}{%
  \setlength{\itemsep}{0pt}\setlength{\parskip}{0pt}}
\setcounter{secnumdepth}{-\maxdimen} % remove section numbering
\ifLuaTeX
  \usepackage{selnolig}  % disable illegal ligatures
\fi
\IfFileExists{bookmark.sty}{\usepackage{bookmark}}{\usepackage{hyperref}}
\IfFileExists{xurl.sty}{\usepackage{xurl}}{} % add URL line breaks if available
\urlstyle{same}
\hypersetup{
  pdftitle={Predictive Analytics},
  pdfauthor={Juhyun},
  hidelinks,
  pdfcreator={LaTeX via pandoc}}

\title{Predictive Analytics}
\author{Juhyun}
\date{2024-11-22}

\begin{document}
\maketitle

\hypertarget{libraries}{%
\subsubsection{Libraries}\label{libraries}}

\begin{Shaded}
\begin{Highlighting}[]
\FunctionTok{library}\NormalTok{(dplyr)}
\end{Highlighting}
\end{Shaded}

\begin{verbatim}
## 
## 다음의 패키지를 부착합니다: 'dplyr'
\end{verbatim}

\begin{verbatim}
## The following objects are masked from 'package:stats':
## 
##     filter, lag
\end{verbatim}

\begin{verbatim}
## The following objects are masked from 'package:base':
## 
##     intersect, setdiff, setequal, union
\end{verbatim}

\begin{Shaded}
\begin{Highlighting}[]
\FunctionTok{library}\NormalTok{(ggplot2)}
\FunctionTok{library}\NormalTok{(tidyverse)}
\end{Highlighting}
\end{Shaded}

\begin{verbatim}
## -- Attaching core tidyverse packages ------------------------ tidyverse 2.0.0 --
## v forcats   1.0.0     v stringr   1.5.1
## v lubridate 1.9.3     v tibble    3.2.1
## v purrr     1.0.2     v tidyr     1.3.1
## v readr     2.1.5
\end{verbatim}

\begin{verbatim}
## -- Conflicts ------------------------------------------ tidyverse_conflicts() --
## x dplyr::filter() masks stats::filter()
## x dplyr::lag()    masks stats::lag()
## i Use the conflicted package (<http://conflicted.r-lib.org/>) to force all conflicts to become errors
\end{verbatim}

\begin{Shaded}
\begin{Highlighting}[]
\FunctionTok{library}\NormalTok{(modelr)}
\FunctionTok{library}\NormalTok{(boot)}
\end{Highlighting}
\end{Shaded}

\hypertarget{import-dataset}{%
\subsubsection{Import Dataset}\label{import-dataset}}

\begin{Shaded}
\begin{Highlighting}[]
\NormalTok{data }\OtherTok{\textless{}{-}} \FunctionTok{read.csv}\NormalTok{(}\StringTok{"climate\_change\_impact\_on\_agriculture\_2024.csv"}\NormalTok{)}
\end{Highlighting}
\end{Shaded}

\hypertarget{preparation-and-cleaning-the-data}{%
\subsubsection{Preparation and cleaning the
data}\label{preparation-and-cleaning-the-data}}

\begin{Shaded}
\begin{Highlighting}[]
\CommentTok{\# mutate continent columns}

\NormalTok{country\_to\_continent }\OtherTok{\textless{}{-}} \FunctionTok{data.frame}\NormalTok{(}
  \AttributeTok{Country =} \FunctionTok{c}\NormalTok{(}\StringTok{"Argentina"}\NormalTok{, }\StringTok{"Australia"}\NormalTok{, }\StringTok{"Brazil"}\NormalTok{,}\StringTok{"Canada"}\NormalTok{,}\StringTok{"China"}\NormalTok{,}\StringTok{"France"}\NormalTok{,}
              \StringTok{"India"}\NormalTok{,}\StringTok{"Nigeria"}\NormalTok{, }\StringTok{"Russia"}\NormalTok{,}\StringTok{"USA"}\NormalTok{),}
  \AttributeTok{Continent =} \FunctionTok{c}\NormalTok{(}\StringTok{"South America"}\NormalTok{, }\StringTok{"Oceania"}\NormalTok{, }\StringTok{"South America"}\NormalTok{, }
                \StringTok{"North America"}\NormalTok{, }\StringTok{"East Asia"}\NormalTok{, }\StringTok{"Europe"}\NormalTok{, }
                \StringTok{"South Asia"}\NormalTok{, }\StringTok{"Africa"}\NormalTok{,}\StringTok{"Eurasia"}\NormalTok{,}\StringTok{"North America"}\NormalTok{)}
\NormalTok{)}

\NormalTok{data\_with\_continent }\OtherTok{\textless{}{-}}\NormalTok{ data }\SpecialCharTok{\%\textgreater{}\%}
  \FunctionTok{left\_join}\NormalTok{(country\_to\_continent, }\AttributeTok{by =} \StringTok{"Country"}\NormalTok{) }

\NormalTok{data }\OtherTok{\textless{}{-}}\NormalTok{ data\_with\_continent }\SpecialCharTok{\%\textgreater{}\%}
  \FunctionTok{select}\NormalTok{(Year, Country, Continent, Region, }\FunctionTok{everything}\NormalTok{())}
\end{Highlighting}
\end{Shaded}

\begin{Shaded}
\begin{Highlighting}[]
\NormalTok{aggregated\_data }\OtherTok{\textless{}{-}}\NormalTok{ data }\SpecialCharTok{\%\textgreater{}\%}
  \FunctionTok{group\_by}\NormalTok{(Year, Continent) }\SpecialCharTok{\%\textgreater{}\%}
  \FunctionTok{summarize}\NormalTok{(}
    \AttributeTok{avg\_crop\_yield =} 
      \FunctionTok{mean}\NormalTok{(Crop\_Yield\_MT\_per\_HA, }\AttributeTok{na.rm =} \ConstantTok{TRUE}\NormalTok{),}
    \AttributeTok{avg\_extreme\_weather\_events =}
      \FunctionTok{mean}\NormalTok{(Extreme\_Weather\_Events, }\AttributeTok{na.rm=} \ConstantTok{TRUE}\NormalTok{),}
    \AttributeTok{avg\_temp\_c =} 
      \FunctionTok{mean}\NormalTok{(Average\_Temperature\_C, }\AttributeTok{na.rm =} \ConstantTok{TRUE}\NormalTok{),}
    \AttributeTok{avg\_total\_precipitation\_mm =} 
      \FunctionTok{mean}\NormalTok{(Total\_Precipitation\_mm, }\AttributeTok{na.rm =} \ConstantTok{TRUE}\NormalTok{),}
    \AttributeTok{avg\_co2\_emissions\_mt =} 
      \FunctionTok{mean}\NormalTok{(CO2\_Emissions\_MT, }\AttributeTok{na.rm =}\ConstantTok{TRUE}\NormalTok{),}
    \AttributeTok{avg\_pesticide\_use\_kg\_per\_ha =} 
      \FunctionTok{mean}\NormalTok{(Pesticide\_Use\_KG\_per\_HA, }\AttributeTok{na.rm=}\ConstantTok{TRUE}\NormalTok{),}
    \AttributeTok{avg\_fertilizer\_use\_kg\_per\_ha =} 
      \FunctionTok{mean}\NormalTok{(Fertilizer\_Use\_KG\_per\_HA, }\AttributeTok{na.rm =}\ConstantTok{TRUE}\NormalTok{),}
    \AttributeTok{avg\_soil\_health\_index =} 
      \FunctionTok{mean}\NormalTok{(Soil\_Health\_Index, }\AttributeTok{na.rm=}\ConstantTok{TRUE}\NormalTok{),}
    \AttributeTok{avg\_economic\_impact\_million\_usd =} 
      \FunctionTok{mean}\NormalTok{(Economic\_Impact\_Million\_USD, }\AttributeTok{na.rm =} \ConstantTok{TRUE}\NormalTok{)}
\NormalTok{  ) }\SpecialCharTok{\%\textgreater{}\%}
  \FunctionTok{ungroup}\NormalTok{() }
\end{Highlighting}
\end{Shaded}

\begin{verbatim}
## `summarise()` has grouped output by 'Year'. You can override using the
## `.groups` argument.
\end{verbatim}

\begin{Shaded}
\begin{Highlighting}[]
\NormalTok{data }\OtherTok{\textless{}{-}}\NormalTok{ data }\SpecialCharTok{\%\textgreater{}\%}
  \FunctionTok{left\_join}\NormalTok{(aggregated\_data, }\AttributeTok{by =} \FunctionTok{c}\NormalTok{(}\StringTok{"Year"}\NormalTok{, }\StringTok{"Continent"}\NormalTok{))}

\NormalTok{data\_constracted }\OtherTok{\textless{}{-}}\NormalTok{ data }\SpecialCharTok{\%\textgreater{}\%}
  \FunctionTok{select}\NormalTok{(}\SpecialCharTok{{-}}\FunctionTok{c}\NormalTok{(}\DecValTok{6}\SpecialCharTok{:}\DecValTok{9}\NormalTok{, }\DecValTok{12}\SpecialCharTok{:}\DecValTok{14}\NormalTok{, }\DecValTok{16}\NormalTok{))}
\end{Highlighting}
\end{Shaded}

\begin{Shaded}
\begin{Highlighting}[]
\FunctionTok{set.seed}\NormalTok{(}\DecValTok{10000}\NormalTok{)}

\NormalTok{train\_df }\OtherTok{\textless{}{-}}\NormalTok{ data }\SpecialCharTok{\%\textgreater{}\%} \FunctionTok{sample\_frac}\NormalTok{(}\FloatTok{0.7}\NormalTok{)}

\NormalTok{test\_df }\OtherTok{\textless{}{-}} \FunctionTok{anti\_join}\NormalTok{(data, train\_df)}
\end{Highlighting}
\end{Shaded}

\begin{verbatim}
## Joining with `by = join_by(Year, Country, Continent, Region, Crop_Type,
## Average_Temperature_C, Total_Precipitation_mm, CO2_Emissions_MT,
## Crop_Yield_MT_per_HA, Extreme_Weather_Events, Irrigation_Access_.,
## Pesticide_Use_KG_per_HA, Fertilizer_Use_KG_per_HA, Soil_Health_Index,
## Adaptation_Strategies, Economic_Impact_Million_USD, avg_crop_yield,
## avg_extreme_weather_events, avg_temp_c, avg_total_precipitation_mm,
## avg_co2_emissions_mt, avg_pesticide_use_kg_per_ha,
## avg_fertilizer_use_kg_per_ha, avg_soil_health_index,
## avg_economic_impact_million_usd)`
\end{verbatim}

\begin{Shaded}
\begin{Highlighting}[]
\NormalTok{train\_df }\OtherTok{\textless{}{-}}\NormalTok{ train\_df }\SpecialCharTok{\%\textgreater{}\%}
  \FunctionTok{mutate}\NormalTok{(}
    \AttributeTok{continent =} \FunctionTok{as.factor}\NormalTok{(Continent),}
    \AttributeTok{crop\_type =} \FunctionTok{as.factor}\NormalTok{(Crop\_Type),}
    \AttributeTok{average\_temperature\_c =} \FunctionTok{as.numeric}\NormalTok{(Average\_Temperature\_C),}
    \AttributeTok{extreme\_weather\_events =} \FunctionTok{as.numeric}\NormalTok{(Extreme\_Weather\_Events), }
    \AttributeTok{avg\_crop\_yield =} \FunctionTok{as.numeric}\NormalTok{(avg\_crop\_yield) }
\NormalTok{  )}
\end{Highlighting}
\end{Shaded}

\begin{Shaded}
\begin{Highlighting}[]
\NormalTok{train\_df }\SpecialCharTok{\%\textgreater{}\%}
  \FunctionTok{summarize}\NormalTok{(}
    \AttributeTok{total =} \FunctionTok{n}\NormalTok{(),}
    \AttributeTok{missing =} \FunctionTok{sum}\NormalTok{(}\FunctionTok{is.na}\NormalTok{(avg\_crop\_yield)),}
    \AttributeTok{fraction\_missing =}\NormalTok{ missing }\SpecialCharTok{/}\NormalTok{ total}
\NormalTok{  )}
\end{Highlighting}
\end{Shaded}

\begin{verbatim}
##   total missing fraction_missing
## 1  7000       0                0
\end{verbatim}

\begin{Shaded}
\begin{Highlighting}[]
\NormalTok{train\_df }\OtherTok{\textless{}{-}}\NormalTok{ train\_df }\SpecialCharTok{\%\textgreater{}\%}
  \FunctionTok{mutate}\NormalTok{(}\AttributeTok{avg\_crop\_yield =} \FunctionTok{if\_else}\NormalTok{(}\FunctionTok{is.na}\NormalTok{(avg\_crop\_yield), }
                                  \FunctionTok{mean}\NormalTok{(avg\_crop\_yield, }\AttributeTok{na.rm =} \ConstantTok{TRUE}\NormalTok{), }
\NormalTok{                                  avg\_crop\_yield))}
\end{Highlighting}
\end{Shaded}

\hypertarget{crop_yield}{%
\paragraph{crop\_yield}\label{crop_yield}}

\begin{Shaded}
\begin{Highlighting}[]
\FunctionTok{library}\NormalTok{(randomForest)}
\end{Highlighting}
\end{Shaded}

\begin{verbatim}
## Warning: 패키지 'randomForest'는 R 버전 4.4.2에서 작성되었습니다
\end{verbatim}

\begin{verbatim}
## randomForest 4.7-1.2
\end{verbatim}

\begin{verbatim}
## Type rfNews() to see new features/changes/bug fixes.
\end{verbatim}

\begin{verbatim}
## 
## 다음의 패키지를 부착합니다: 'randomForest'
\end{verbatim}

\begin{verbatim}
## The following object is masked from 'package:ggplot2':
## 
##     margin
\end{verbatim}

\begin{verbatim}
## The following object is masked from 'package:dplyr':
## 
##     combine
\end{verbatim}

\begin{Shaded}
\begin{Highlighting}[]
\CommentTok{\# Train a random forest model}
\NormalTok{rf\_model }\OtherTok{\textless{}{-}} \FunctionTok{randomForest}\NormalTok{(}
\NormalTok{  avg\_crop\_yield }\SpecialCharTok{\textasciitilde{}}\NormalTok{ average\_temperature\_c }\SpecialCharTok{+}\NormalTok{ extreme\_weather\_events }\SpecialCharTok{+} 
\NormalTok{    Total\_Precipitation\_mm }\SpecialCharTok{+}\NormalTok{ continent }\SpecialCharTok{+}\NormalTok{ crop\_type,}
  \AttributeTok{data =}\NormalTok{ train\_df,}
  \AttributeTok{ntree =} \DecValTok{100}\NormalTok{, }\CommentTok{\# Number of trees}
  \AttributeTok{mtry =} \DecValTok{2}\NormalTok{,    }\CommentTok{\# Number of variables randomly sampled}
  \AttributeTok{importance =} \ConstantTok{TRUE}
\NormalTok{)}

\CommentTok{\# View model summary}
\FunctionTok{print}\NormalTok{(rf\_model)}
\end{Highlighting}
\end{Shaded}

\begin{verbatim}
## 
## Call:
##  randomForest(formula = avg_crop_yield ~ average_temperature_c +      extreme_weather_events + Total_Precipitation_mm + continent +      crop_type, data = train_df, ntree = 100, mtry = 2, importance = TRUE) 
##                Type of random forest: regression
##                      Number of trees: 100
## No. of variables tried at each split: 2
## 
##           Mean of squared residuals: 0.03292584
##                     % Var explained: -4.86
\end{verbatim}

\begin{Shaded}
\begin{Highlighting}[]
\CommentTok{\# Predict on the test dataset}
\NormalTok{test\_df }\OtherTok{\textless{}{-}}\NormalTok{ test\_df }\SpecialCharTok{\%\textgreater{}\%}
  \FunctionTok{mutate}\NormalTok{(}
    \AttributeTok{continent =} \FunctionTok{as.factor}\NormalTok{(Continent),}
    \AttributeTok{crop\_type =} \FunctionTok{as.factor}\NormalTok{(Crop\_Type),}
    \AttributeTok{average\_temperature\_c =} \FunctionTok{as.numeric}\NormalTok{(Average\_Temperature\_C),}
    \AttributeTok{extreme\_weather\_events =} \FunctionTok{as.numeric}\NormalTok{(Extreme\_Weather\_Events) }\CommentTok{\# Ensure numeric type}
\NormalTok{  )}
\CommentTok{\# Predict on the test dataset}
\NormalTok{test\_df }\OtherTok{\textless{}{-}}\NormalTok{ test\_df }\SpecialCharTok{\%\textgreater{}\%}
  \FunctionTok{mutate}\NormalTok{(}
    \AttributeTok{predicted\_yield =} \FunctionTok{predict}\NormalTok{(rf\_model, }\AttributeTok{newdata =}\NormalTok{ test\_df)}
\NormalTok{  )}

\CommentTok{\# Calculate Mean Absolute Error (MAE)}
\NormalTok{mae }\OtherTok{\textless{}{-}} \FunctionTok{mean}\NormalTok{(}\FunctionTok{abs}\NormalTok{(test\_df}\SpecialCharTok{$}\NormalTok{predicted\_yield }\SpecialCharTok{{-}}\NormalTok{ test\_df}\SpecialCharTok{$}\NormalTok{avg\_crop\_yield))}
\FunctionTok{print}\NormalTok{(}\FunctionTok{paste}\NormalTok{(}\StringTok{"Mean Absolute Error:"}\NormalTok{, mae))}
\end{Highlighting}
\end{Shaded}

\begin{verbatim}
## [1] "Mean Absolute Error: 0.137131315371807"
\end{verbatim}

\begin{Shaded}
\begin{Highlighting}[]
\FunctionTok{head}\NormalTok{(test\_df }\SpecialCharTok{\%\textgreater{}\%}
  \FunctionTok{select}\NormalTok{(Continent, Crop\_Type, avg\_crop\_yield, predicted\_yield) }\SpecialCharTok{\%\textgreater{}\%}
  \FunctionTok{arrange}\NormalTok{(predicted\_yield))}
\end{Highlighting}
\end{Shaded}

\begin{verbatim}
##   Continent  Crop_Type avg_crop_yield predicted_yield
## 1   Eurasia     Coffee       2.097615        2.005410
## 2 East Asia Vegetables       1.859909        2.029474
## 3   Eurasia   Soybeans       2.137800        2.043336
## 4   Eurasia       Corn       2.245636        2.075565
## 5   Eurasia       Corn       2.149656        2.078574
## 6   Eurasia Vegetables       1.915600        2.081742
\end{verbatim}

\hypertarget{economic_impact}{%
\section{economic\_impact}\label{economic_impact}}

\begin{Shaded}
\begin{Highlighting}[]
\CommentTok{\# Train a random forest model to predict economic impact}
\NormalTok{rf\_model }\OtherTok{\textless{}{-}} \FunctionTok{randomForest}\NormalTok{(}
\NormalTok{  avg\_economic\_impact\_million\_usd }\SpecialCharTok{\textasciitilde{}}\NormalTok{ average\_temperature\_c }\SpecialCharTok{+}\NormalTok{ extreme\_weather\_events }\SpecialCharTok{+} 
\NormalTok{    avg\_total\_precipitation\_mm }\SpecialCharTok{+}\NormalTok{ continent }\SpecialCharTok{+}\NormalTok{ crop\_type }\SpecialCharTok{+}\NormalTok{ avg\_crop\_yield,}
  \AttributeTok{data =}\NormalTok{ train\_df,}
  \AttributeTok{ntree =} \DecValTok{100}\NormalTok{, }\CommentTok{\# Number of trees}
  \AttributeTok{mtry =} \DecValTok{2}\NormalTok{,    }\CommentTok{\# Number of variables randomly sampled}
  \AttributeTok{importance =} \ConstantTok{TRUE}
\NormalTok{)}

\CommentTok{\# View model summary}
\FunctionTok{print}\NormalTok{(rf\_model)}
\end{Highlighting}
\end{Shaded}

\begin{verbatim}
## 
## Call:
##  randomForest(formula = avg_economic_impact_million_usd ~ average_temperature_c +      extreme_weather_events + avg_total_precipitation_mm + continent +      crop_type + avg_crop_yield, data = train_df, ntree = 100,      mtry = 2, importance = TRUE) 
##                Type of random forest: regression
##                      Number of trees: 100
## No. of variables tried at each split: 2
## 
##           Mean of squared residuals: 152.5297
##                     % Var explained: 97.32
\end{verbatim}

\begin{Shaded}
\begin{Highlighting}[]
\CommentTok{\#Predict on the test dataset}
\NormalTok{test\_df }\OtherTok{\textless{}{-}}\NormalTok{ test\_df }\SpecialCharTok{\%\textgreater{}\%}
  \FunctionTok{mutate}\NormalTok{(}
    \AttributeTok{predicted\_economic\_impact =} \FunctionTok{predict}\NormalTok{(rf\_model, }\AttributeTok{newdata =}\NormalTok{ test\_df)}
\NormalTok{  )}

\CommentTok{\# Calculate Mean Absolute Error (MAE)}
\NormalTok{mae }\OtherTok{\textless{}{-}} \FunctionTok{mean}\NormalTok{(}\FunctionTok{abs}\NormalTok{(test\_df}\SpecialCharTok{$}\NormalTok{predicted\_economic\_impact }\SpecialCharTok{{-}}\NormalTok{ test\_df}\SpecialCharTok{$}\NormalTok{avg\_economic\_impact\_million\_usd))}
\FunctionTok{print}\NormalTok{(}\FunctionTok{paste}\NormalTok{(}\StringTok{"Mean Absolute Error:"}\NormalTok{, mae))}
\end{Highlighting}
\end{Shaded}

\begin{verbatim}
## [1] "Mean Absolute Error: 7.85622101969733"
\end{verbatim}

\begin{Shaded}
\begin{Highlighting}[]
\FunctionTok{head}\NormalTok{(test\_df }\SpecialCharTok{\%\textgreater{}\%}
  \FunctionTok{select}\NormalTok{(Continent, Crop\_Type, avg\_economic\_impact\_million\_usd, predicted\_economic\_impact) }\SpecialCharTok{\%\textgreater{}\%}
  \FunctionTok{arrange}\NormalTok{(predicted\_economic\_impact))}
\end{Highlighting}
\end{Shaded}

\begin{verbatim}
##   Continent Crop_Type avg_economic_impact_million_usd predicted_economic_impact
## 1    Africa    Barley                        454.3550                  469.0314
## 2    Europe    Cotton                        465.3039                  472.3679
## 3    Europe     Wheat                        465.3039                  472.8046
## 4    Africa  Soybeans                        454.3550                  474.0483
## 5    Africa    Cotton                        454.3550                  479.3397
## 6    Europe  Soybeans                        465.3039                  481.3647
\end{verbatim}

\end{document}
